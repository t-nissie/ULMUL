% nishimatsu-ajt.tex
% Manuscript for ULMUL
% Time-stamp: <2010-11-08 09:38:24 takeshi>
%%
\documentclass{ajt}            % run with pdflatex or xelatex

%%% Preloaded packages: color, graphicx, hyperref, calc, amsmath, textcomp
%%% Packages used in this paper
%\usepackage{amssymb}

%%% Verbatim environment
%\usepackage{fancyvrb}
%\CustomVerbatimEnvironment{code}{Verbatim}%
%{numbers=left,xleftmargin=1.5em,baselinestretch=1.069,fontsize=\small}
%\CustomVerbatimEnvironment{codewithoutnum}{Verbatim}%
%{xleftmargin=1.5em,baselinestretch=1.069,fontsize=\small}
%\CustomVerbatimEnvironment{codewithoutnumsmall}{Verbatim}%
%{xleftmargin=1.5em,baselinestretch=1.0,fontsize=\footnotesize}
%\DefineShortVerb{\|}

%%% Mandatory article metadata %%%
\title{How to prepare a document with AJT class}
%\title[Subtitle here]{How to prepare a document with AJT class}
\author{Takeshi Nishimatsu}
\address{Department of Typesetting, \TeX\ University}
\email{t-nissie@imr.tohoku.ac.jp}
\keywords{bbb bbb bbb}
\abstract{aldjf dfkjadlf dlfajdlf dfajdl
aaaaaaa aaaaaaaa aaaaaaaaaa}

\begin{document}

%%% Do not forget to start with \maketitle!
\maketitle

\section{aaa}
aaaaaaaaa aaaaaaaaaaa aaaaaaaaaaa\cite{ruby-mathml}.

\subsection{bbbb bbbb bbbb}
aaaa aaaaa aaaa aaaaaaa aaaaa aaaaaa\cite{ulmul-rubyforge-org}.
aaaa aaaaa aaaa aaaaaaa aaaaa aaaaaa.
aaaa aaaaa aaaa aaaaaaa aaaaa aaaaaa.
aaaa aaaaa aaaa aaaaaaa aaaaa aaaaaa.

\subsection{ccc ccc ccc}
aaaa aaaaa aaaa aaaaaaa aaaaa aaaaaa\cite{Slidy}.
aaaa aaaaa aaaa aaaaaaa aaaaa aaaaaa\cite{ulmul-rubygems}.
aaaa aaaaa aaaa aaaaaaa aaaaa aaaaaa.
aaaa aaaaa aaaa aaaaaaa aaaaa aaaaaa.

\bibliographystyle{nishimatsu-ajt} %original bst
\bibliography{nishimatsu-ajt}
\end{document}
%%%Local variables:
%%%  TeX-PDF-mode: t
%%%End:
