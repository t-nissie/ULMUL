% nishimatsu-ajt.tex
% Manuscript for ULMUL
% Time-stamp: <2011-02-06 20:36:54 takeshi>
%%
\documentclass{ajt}            % run with pdflatex or xelatex

%%% Preloaded packages: color, graphicx, hyperref, calc, amsmath, textcomp
%%% Packages used in this paper
%\usepackage{amssymb}

%%% Verbatim environment
%\usepackage{fancyvrb}
%\CustomVerbatimEnvironment{code}{Verbatim}%
%{numbers=left,xleftmargin=1.5em,baselinestretch=1.069,fontsize=\small}
%\CustomVerbatimEnvironment{codewithoutnum}{Verbatim}%
%{xleftmargin=1.5em,baselinestretch=1.069,fontsize=\small}
%\CustomVerbatimEnvironment{codewithoutnumsmall}{Verbatim}%
%{xleftmargin=1.5em,baselinestretch=1.0,fontsize=\footnotesize}
%\DefineShortVerb{\|}

%%% Mandatory article metadata %%%
\title[Development of a Ultra Lightweight MarkUp Language (ULMUL)]{An implementation of markup language to markup language converter}
%\title[Subtitle here]{How to prepare a document with AJT class}
\author{Takeshi Nishimatsu}
\address{Institute for Materials Research (IMR), Tohoku University, Sendai 980-8577, JAPAN}
\email{t-nissie@imr.tohoku.ac.jp}
\keywords{XHTML, HTML5, MathML, presentation, HTML Slidy, Firefox}
\abstract{``ULMUL'' is an original Ultra Lightweight MarkUp Language.
ULMUL texts can be converted into HTML5 with ``ulmul2html5'' command
and into XHTML with ``ulmul2xhtml'' command.
TeX style equations are converted into MathML.
ULMUL is written in Ruby.
}


\begin{document}
\maketitle

\section{Introduction}
There are many markup languages for the pretty typesetting.
For example,
HTML is one of the most famous markup languages,
Ruby Document format (RD)\cite{RDtool} and
RDoc\cite{RDoc} are widely used in Ruby communities, and
ReVIEW is designed for the digital publishing\cite{ReVIEW}.
Here, I introduce an original Ultra Lightweight MarkUp Language, named ULMUL,
which I am developing for the pretty typesetting and presentations
with mathematical equations.

\section{Features}
ULMUL only has simple and low functions.
ULMUL texts should be written in the UTF-8 encodeing.
aaaa aaaaa aaaa aaaaaaa aaaaa aaaaaa\cite{Slidy}.
aaaa aaaaa aaaa aaaaaaa aaaaa aaaaaa\cite{ulmul-rubygems}.
aaaa aaaaa aaaa aaaaaaa aaaaa aaaaaa.
aaaa aaaaa aaaa aaaaaaa aaaaa aaaaaa.



\section{Implementation}
Currently, the main library file of ULMUL is about 300 lines of Ruby code.
The MathML library\cite{ruby-mathml} is employed for the TeX to MathML Processing.
A finite state machine is applied for parsing ULMUL text.
I distribute ULMUL as free software under the conditions described in
the GNU General Public License (the "GPL").
Because I prepare ``gem'' packages of every versions of ULMUL from {\tt RubyGems.org},
You can install ULMUL merely with ``{\tt sudo gem install ulmul}''.

\section{Future of ULMUL}
``ulmul2latex'' command.

\bibliographystyle{nishimatsu-ajt} %original bst
\bibliography{nishimatsu-ajt}
\end{document}
%%%Local variables:
%%%  TeX-PDF-mode: t
%%%End:
